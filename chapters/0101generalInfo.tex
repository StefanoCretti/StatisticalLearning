\chapter{General course information}
  \section{Textbooks}
    \begin{itemize}
      \item James et al (2021), Introduction to statistical learning in R, 2nd edition. (Book that is used as guidline for the course, but further concepts will be added during the lectures)
      \item Hastie et al (2001), Elements of statistical learning. (More advanced book for those who want to study the subject more in depth)
    \end{itemize}

  \section{Assessment}
    \begin{itemize}
      \item Three homework tasks during the course (Uploaded on moodle, two weeks of time for each, more practical and mainly focused on applying methods to some data. If done well they will add 2 points to the written exam score)
      \item Final written exam (More theoretical but still connected to the practical part, for instance by commenting on analysis output)
    \end{itemize}

  \section{Topics}
    \begin{itemize}
      \item Linear regression (Gauss, 1800) (Assumed to be already known from Statistical Learning 1)
      \item Linear discriminant analysis, LDA (Fisher, 1936) (Later extended to quadratic discriminant analysis, QDA)
      \item Logistic regression (1940s)
      \item Generalized linear models (Nelder and Wedderburn, 1972)
      \item Classification and regression trees (Breiman and Freidman, 1980s) (First introduction of computer intensive methods)
      \item Machine learning (1990s): support vector machines, neural networks/deep learning, unsupervised learning (clustering, PCA)
      \item Individual methods: theory, details, implementation ...
      \item General concept: model selection, inference, prediction ...
    \end{itemize}